\section{Questions from other books or sources.}

\begin{qanda}
  \Q You are told a coin has 60\% of probability of landing heads.
  If I gave you 1 to 1 odds, would you want to play? 
  How would you be in order to maximise your winnings and minimise risk in a long term betting strategy?

  \A
  As the expectation of the game is positive ($-0.4 + 0.6=0.2$), a strategy maximising the expected value suggests us to play.
  By the law of large numbers, the more games are played, the closer the average winnings will be to the expected value.
  Under this perspective, the question of how much to bet appears to be irrelevant.

  However, the CLT only holds if we \emph{can} play an infinite number of games.
  This may not be the case, if, for example we bet all our money in the first game and lose.
  To minimise the risk of ruin, knowing we can play an infinite number of games, we should bet an infinitesimal amount of our wealth infinitely many times.

  The answer to this question may be more interesting if the game can only be played $N$ times.
  Let $X_n$ be the amount of money we have at time $n$.
  Consider two strategies:
  \begin{itemize}
      \item[\textbf{F}] Betting a fixed amount of money $M$ at each game, if $X_n \geq M$.
      \item[\textbf{Q}] Betting a fraction $\alpha$ of our wealth at each game.
  \end{itemize}
  The first strategy incurs a risk of ruin, while the second one doesn't. Intuitively, this is more sound, so let us examine it.
  The profits after $T$ games can be explicitly written as:
  \[
      X_T = X_0 \prod_{n=1}^T (1 + \alpha(2H_{n} - 1)),
  \]
  where $H_n$ is a Bernoulli random variable with probability $0.6$ indicating whether the $n$-th coin flip was heads. 
  It is natural to consider the logarithm, to transform the product into a sum.
  Let us define $L_t := \log X_t$, and $B_t =\log(1 + \alpha(2H_{n} - 1)) $ so that:
  \[
  L_T = L_0 + \sum_{n=1}^T \log(1 + \alpha(2H_{n} - 1)) = L_0 + \sum_{n=1}^T B_n.
  \]
  We have a sum of independent random variables, so the CLT applies.
  We maximise the expected value of $L_T$ by maximising the expected value of $B_n$:
  \[
      \mathbf{E}[B_n] = \log(1 + \alpha) 0.6 + \log(1 - \alpha) 0.4. 
  \]
  Taking derivatives and setting them to 0 yields:
  \[
  \frac{0.6}{1+\alpha} = \frac{0.4}{1-\alpha} \implies
  6 - 6\alpha = 4 + 4\alpha \implies \alpha = 0.2.
  \]
  We can thus maximise the growth rate of our wealth by betting 20\% of our wealth at each game.
\end{qanda}
