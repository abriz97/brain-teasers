\section{Probability Theory}

\subsection{Basic probability and set operations}
 
\begin{qanda} % Coin Toss Game
    \Q
  Two gamblers are playing a coin toss game.
  Gambler $A$ has $(n+1)$ fair coins; $B$ has $n$ fair coins.
  What is the probability that $A$ will have more heads than $B$ if both flip all their coins?

  \A
  Let us define $N^P_ i$ as the number of heads obtained by player $P$ after $i$ coin tosses.
  We can compute the required probability by conditioning on what would have happened right before the last coin toss from $A$.
  \[
      \mathbf{P}( N^A_{n+1} > N^B_n ) = 
      \mathbf{P}( N^A_n} > N^B_n) +
\mathbf{P}( N^A_n} = N^B_n}) \cdot \frac{1}{2} +
      \mathbf{P}( N^A_n} < N^B_n}) \cdot 0.
  \]
  By simmetry, if $p := \mathbf{P}( N^A_n} < N^B_n})$, we have: 
  \[
      \mathbf{P}( N^A_{n+1} > N^B_n ) = 
      p +
      (1-2p) \cdot \frac{1}{2} = \frac{1}{2}  + p - p = \frac{1}{2}.
  \]
  \emph{I did not expect the answer to be so simple!}

  % Clearly, if $A$ and $B$ had the same number of coins, the probability of $A$ having more heads than $B$ would be $\frac{1 - \mathbf{P}(N^A_n} = N^B_n})}{2$.
  % The probability of a draw can be calculated as \ldots 
\end{qanda}

\begin{qanda} % Card Game
    \Q
    A casino offers a simple card game.
    There are 52 cards in a deck with 4 cards for each value.
    Each time the cards are thoroughly shuffled.
    You pick up a card from the deck and the dealer picks another one without replacement. 
    If you have a larger number, you win; if the numbers are equal or yours is smaller, the house wins. What is the probability of you winning?

    \A
    There are probably multiple ways to think about this. 
    The first that comes to mind consists in calculating the probability $p$ of the event where the two drawn cards have the same value. Then, the desired probability would be $\frac{1-p}{2}$.
    The calculations are easy! $p = \frac{3}{51}$, so the desired probability is $\frac{48}{102} = \frac{8}{17}$.
\end{qanda}

\begin{qanda} % Drunk Passenger
    \Q
    A line of 100 airline passengers are waiting to board a plane.
    They each hold a ticket to one of the 100 seats on that flight.
    For convenience, let's say that the $n$-th passenger in line has a ticket for the seat number $n$.
    Being drunk, the first person in line picks a random seat (equally likely for each seat). 
    All of the other passengers are sober, and will go to their proper seats unless it is already occupied; in that case, they will randomly choose a free seat.
    You are person number 100. What is the probability that you end up in your seat (i.\,e.\ seat number 100)?

    \A
    I know the problem, but let us try to analyse it in line with the techniques learnt until now. There is a feeling for which seats number 2, \ldots, 99 are equivalent, while seats 1 and 100 are \textquote{special}.
    This suggests us to look at two complementary events: the event where seat 1 is taken before seat 100, and the event where seat 100 is taken before seat 1.
    Note both events have equal probability (by simmetry).
    If seat 1 is taken before seat 100, say by passenger $k$, this means that the following passengers will take their own seats, including me.
    As such, the probability of sitting at the correct place is $\frac{1}{2}$.
\end{qanda}

\begin{qanda} % N points on a circle
  \Q
  Given $N$ points drawn randomly on the circumference of a circle, what is the probability that they are all within a semicircle?
\end{qanda}

\subsection{Combinatorial analysis}

\begin{qanda} % Poker hands
  \Q
  We are playing poker with hands of 5 cards.
  What are the probabilities of getting hands with four-of-a-kind? 
  Hands with a full house?
  Hands with two pairs?

  \A
  \[
  \mathbf{P}(\text{Four-of-a-kind}) =  
  \left( 1 \cdot \frac{3}{51} \cdot \frac{2}{50} \cdot \frac{1}{49}  \cdot 1 \right) \cdot \binom{5}{1} = \frac{3}{51\cdot 5\cdot 49} \approx 0.00024.
  \]

  For a full-house, we need to chose two cards out of 13, but order matters!
  Indeed one card will be repeated three times, while the other will be repeated twice.
  \[
  \mathbf{P}(\text{Full-house}) =  \frac{52 \cdot 48 \cdot 3 \cdot  3 \cdot  2 \cdot \binom{5}{2} }{52 \cdot 51 \cdot  50 \cdot  49 \cdot  48} = \frac{180}{51\cdot 50\cdot 49} \approx 0.0014.
  \]
  Hands with two pairs: % should be 0.047
  \[
  \mathbf{P}( \text{ Two pairs} )  = \frac{52 \cdot 48 \cdot  3 \cdot  3 \cdot 44 \cdot \binom{5}{1}}{52! / 47!} = \frac{9 \cdot  44}{ 51 \cdot 10 \cdot 49} \approx 0.0033.
  \]

  \emph{The last seems wrong. In general, it might be wiser to count the numerator and denominators separately, combinatorially.}
\end{qanda}

\begin{qanda} % Hopping rabbit
  \Q
  A rabbit sits at the bottom of a staircase with $n$ stairs.
  The rabbit can hop up only one or two stairs at a time. How many different ways are there for the rabbit to ascend to the top of the stairs?

  \A
  This is a dynamic programming problem.
  Let $f(n)$ be the number of ways to reach the $n$-th step.
  Clearly, $f(1)=1$ and $f(2) = 2$.
  For $n \geq 3$, we have $f(n) = f(n-1) + f(n-2)$, and we can see this corresponds to the Fibonacci sequence. 
  To calculate an expression for $f(n)$, we can use linear algebra.
  \[
  \begin{pmatrix} f(n+2) \\ f(n+1) \end{pmatrix} 
  = 
  \begin{pmatrix}
    1 & 1  \\
    1 & 0 
  \end{pmatrix}^1
  \begin{pmatrix} f(n+1) \\ f(n) \end{pmatrix}  =
  \begin{pmatrix}
    1 & 1  \\
    1 & 0 
\end{pmatrix}^{n}
  \begin{pmatrix} f(2) \\ f(1) \end{pmatrix} 
  \]
  We can then diagonalise the matrix to obtain an expression for $f(n+2)$.
  The eigenvalues solve $0=-(1-\lambda)\lambda - 1 = \lambda^2 - \lambda - 1$, so that $\lambda_{1,2} = \frac{1 \pm \sqrt{5}}{2}$.
  We can diagonalise the matrix as $A = P D P^{-1}$, where $D$ is a diagonal matrix with the eigenvalues on the diagonal, which will then yield a polynomial expression in terms of the two roots $\lambda_{1,2}$.
\end{qanda}

\begin{qanda} % Screwy pirates 2
    \Q 
    There are 11 pirates on a ship, with a loot they just won.
    To protect their hard-won treasure, they gather together to put all the loot in a safe.
    Still being a democratic bunch, they decide that only a majority - any majority - of them ($\geq 6$) together can open the safe.
    To access the treasure, every lock needs to be opened. 
    Each lock can have multiple keys; but each key only opens one lock. The locksmith can give more than one key to each pirate.

    What is the smallest number of locks needed?
    And how many many keys must each pirate carry?

    \A 
    We know that any group of 6 pirates can access the treasure.
    It feels like there need to be at least 6 locks. 
    If there were only 5 locks, then only 5 keys would be used, meaning a subset of 5 pirates could access the locks.

    We want all groups of 5 pirates $\binom{11}{5}$ not to be able to open the locks, and all groups of 6 pirates $\binom{11}{6}$ to be able to.
    The two numbers equal to $77$.

    For simplicity, let us consider the case in which all pirates have the same number of keys, $k$, each opening a separate lock, from a total of $l$.
    In this scenario, each 5-pirates team would have $5k$ keys, while each 6-pirates team would have $6k$ keys.

    \emph{need to think about this more carefully, but sounds interesting.}
\end{qanda}

\begin{qanda} % Chess tournament
    \Q 
    A chess tournament has $2^n$ players with skills $1>2> \dots > 2^n$. It is organized as a knockout tournament, so that after each round only the winner proceeds to the next round. 
    Except for the final, opponents in each round are drawn at random. 
    Let's also assume that when two players meet in a game, the player with better skills always wins.
    What's the probability that players 1 and 2 will meet in the final?

    \A 
    Upon thinking about it, the answer should be $\frac{1}{2}$ (or close to). The random allocation after every turn doesn't really matter, we can instead consider a \textquote{typical} tournament table, where the initial \textquote{positions} are randomly allocated.
    1 and 2 will meet in the final if they are spawned into two different halves of the table. 
    This happens with probability $1 \cdot \frac{2^{n-1}}{2^n - 1}$.

    % As player 1 will always win, she will always reach the final. Player 2 will reach the final if and only if she doesn't meet player 1 beforehand.
    % We are then interested in computing the probability that the two players meet in the other rounds, so let us denote by $p_j$ the probability of meeting in round $j$ (the final being round $n$). Let us also denote by $p_{j|j-1}$ the probability of meeting in round $j$ given that they didn't meet in round $j-1$.
    % \[
    %     p_n = 1 - \sum_{k=1}^{n-1} p_k =
    %     1 - p_1 - \sum_{k=1}^{n-2} p_{k} p_{k+1|k}
    % \]
    % I am waffling, let us start with computations instead:
    % \begin{align*}
    %     p_1 &= \frac{2^n - 2}{2^n} = \frac{2^{n-1} - 1}{2^{n-1}} = 1 - \frac{1}{2^{n-1}}, \\
    %     p_2 &= (1 - p_1) \cdot  \frac{1}{2^{n-1}}  \\
    %     p_3 &= (1 - p_2)(1-p_1) \cdot  \frac{1}{2^{n-2}} 
    % \end{align*}

    % There must be a simpler solution, it feels like it should be 1/2.

\end{qanda}

\begin{qanda} % Application letters
    \Q 
    You are sending job applications to 5 firms: A, B, C, D, E.
    You have 5 envelopes on the table for each firm, and 5 cover letters personalized for each firm.
    Your friend comes and mixes the letters in each envelope.
    What is the probability that all 5 cover letters are mailed to the wrong firms?

    \A 
    After thinking a little about this, I came to think of the problem in terms of \textquote{cycles}. What I mean can be clarified by noting that every allocation of letters corresponds to the specification of a permutation map:
    \[
    p\colon \{A, B, C, D, E\}\to \{A, B, C, D, E\}.
    \]
    Now, $p$ could be \textquote{composed} of two other permutation maps, for example, one shuffling elements of the set $\{A, B\}$, and the other shuffling the elements of the set $\{C, D, E\}$ \footnote{\textit{ This \textquote{combination} operation is called the direct sum, but this is not relevant.}}.
    For each permutation map, we can define the \textquote{length of the cycle} $c$ as the minimum number of repeated applications of the map to return to the original position, that is $p^c(X) = X$ for all $X$ in the domain.
    There are two cases that are allowed, if we want to avoid sending a cover letter to the correct firm:
    \begin{itemize}
        \item A single cycle of length 5. Let us define the start of the cycle as the letter for firm $A$. There are 4 choices for the second letter, 3 for the third, 2 for the fourth and 1 for the last. This gives $4!$ possibilities.
        \item A cycle of length 3, and one of length 2. Similar as the above, if we had 3 firms selected for the 3-cycle, these would amount to $2!$ firms. The remaining 2 firms would be selected in $1!$ ways.
        The total is then $\binom{5}{3} 2!$
    \end{itemize}
    The required probability amounts to $\frac{4! + \binom{5}{3} 2!}{5!} = \frac{(24 + 20)}{5!} = \frac{11}{30}$.

    \emph{
    The correct solution involved a cute application of the inclusion-exclusion principle, based on the definition of the event $E_i$ as being the event where the $i$-th letter is sent to the correct firm. I think my solution is interesting though, and it's interesting how these two might relate.
    As the number of letter scales up, my solution might be unfeasible because we require the computation of the number of partitions?
    }
\end{qanda}

\begin{qanda} % Birthday problem
    \Q
    How many people do we need in a class to make the probability that two people have the same birthday more than $\frac{1}{2}$?

    \A
    Consider $n$ classmates, and a year of 365 days.
    The probability that no two people share the same birthday is:
    \[
        p_n = \frac{365! / (365-n)! }{365^n} = 
        p_{n-1} \cdot \frac{365-(n-1)}{365}.
    \]
    We can put this in a computer, but can we guess such $n$ from the following recurrence relation?
    The minimum $n$ such that $p_n < \frac{1}{2}$ is 23.
\end{qanda}

\begin{qanda} % 100th digit
  \Q
  What is the 100\tsup{th} digit to the right of the decimal point in the decimal representation of $(1 + \sqrt{2})^{3000}$?

  \A
  We care about the 100\tsup{th} digit, this is equivalent to multiplying by $10^{100}$ and taking modulo 10, then floor (or viceversa).
  The answer $d$ can then be expressed as:
  \begin{align*}
      d &= \lfloor 10^{100} (1 + \sqrt{2})^{300 \cdot  100} \mod 10 \rfloor \\
        &= \lfloor (10 \cdot (1+\sqrt{2})^{300} )^{100} \rfloor \mod 10 \\
        &= \lfloor (10 \cdot (1+\sqrt{2})^{300} )\rfloor^{100} \mod 10 
  \end{align*}

  % This question definitely requires some approximations, so let us use the binomial expansion.
  % \[
  %     (1 + \sqrt{2})^{3000} = 1 + 3000 \sqrt{2} + \binom{3000}{2} 2 + \binom{3000}{3} \sqrt{2}2 + \dots
  % \]
  \textcolor{green}{Not the correct solution: interplay of $(1 \pm \sqrt{2})$.}
\end{qanda}

\begin{qanda} % Cubic of integer
  \Q
  Let $x$ be an integer between $1$ and $10^{12}$, what is the probability that the cubic of $x$ ends in $11$?

  \A
  I am stupid, this is way easier than I thought it was.
  If $x \mod 100 = 11$, this means that $x^3 \mod 10 = 1$.
  There are no integers that cubed end in 1, except of 1
  However, $1^3$ and $11^3$ do not end in 11.
  The probability is then 0. \textit{can compete the proof by showing $(100+n)^3 \mod 100 = n^3 \mod 100.$}
\end{qanda}

\subsection{Nu se}


\begin{qanda}
    \Q An ant is placed in a corner of a cube and cannot move. A spider starts from the opposite corner, and can move along the cube's edges in any direction $(x,y,z)$ with equal probability ($\frac{1}{3}$) . On average, how many steps will the spider need to get to the ant?
    \A 
    First thoughts would be to set up a difference equation, i.\,e.\ a function mapping each vertex to the expected number of steps to the reach the ant.
    There exists some symmetry in the cube vertices, meaning that two vertices which are at the same distance from the ant will have the same expected number of steps. 
    (this means the difference equation is set-up at a \textquote{higher level} than the individual vertices, a group level).
    There are 4 types of vertices: one at distance 0, 3 at distance 1, 3 at distance 2 and 1 at distance 3. We have:
    \begin{align*}
        S(0) &= 0 \\
        3 \cdot S(1) &= 3 + 1 \cdot 0 + 2 \cdot S(2) \\
        3 \cdot S(2) &= 3 + 1 \cdot S(3) + 2 \cdot  S(1) \\
        S(3) &= 1 + S(2).
    \end{align*}
    Substituting the last equation into the previous we obtain $2 S(2) = 4 + 2 S(1)$, which implies $S(1) = 7$, $S(2) = 9$ and $S(3) = 10$. \\
    \emph{My solution is way cleaner than what is on \href{https://stats.stackexchange.com/questions/139384/random-walk-on-the-edges-of-a-cube}{Stack Overflow}}!
\end{qanda}

\begin{qanda}
    \Q Consider you have two random variables, $X$ and $Y$, which are joint normal with zero means, unit variances and correlation of $rho$. Describe/derive distribution of $X$ given $Y = y$.
    \A My first approach uses the definition of conditional probability.
    Using the definition of the multivariate normal in the equation $p(x|y) \propto p(x,y)$ yields:
    \begin{equation*} 
        p(x | y) \propto \exp \left(  -\frac{1}{2} (x, y)  
            \begin{pmatrix} 1 &  \rho\\ \rho & 1 \end{pmatrix}^{-1}
            \begin{pmatrix} x \\ y \end{pmatrix} 
        \right).
        \end{equation*}
        To invert the matrix, we recall the formula for the inverse of a 2x2 matrix:
        \begin{equation*}
            \begin{pmatrix} 1 & \rho \\ \rho & 1 \end{pmatrix}^{-1} = 
            \frac{1}{1 - \rho^2}
            \begin{pmatrix} 1 & -\rho  \\ -\rho& 1 \end{pmatrix}
        \end{equation*}
        Computing the quadratic form of the matrix gives $x^2 + y^2 - 2 \rho x y$.
        This yields:
        \begin{equation*}
            p(x | y) \propto \exp \left(  
                -\frac{(x - \rho y)^2}{2(1-\rho^2)} 
            \right)
        \end{equation*}
        which is proportional to a normal distribution with mean $\rho y$ and variance $1 - \rho^2$.  In conclusion:
        \begin{equation*}
            X | Y = y \sim \text{Normal}(\rho y, 1 - \rho^2)
        \end{equation*}
        \emph{I believe this answer is correct, see \href{https://online.stat.psu.edu/stat414/lesson/21/21.1}{link}. I am not sure there is anything more elegant? In terms of generalisation, we can reduce the general case to this one through linear transformations(?).}
    \end{qanda}


