\section{Basic probability}

\begin{qanda}
    \Q An ant is placed in a corner of a cube and cannot move. A spider starts from the opposite corner, and can move along the cube's edges in any direction $(x,y,z)$ with equal probability ($\frac{1}{3}$) . On average, how many steps will the spider need to get to the ant?
    \A 
    First thoughts would be to set up a difference equation, i.\,e.\ a function mapping each vertex to the expected number of steps to the reach the ant.
    There exists some symmetry in the cube vertices, meaning that two vertices which are at the same distance from the ant will have the same expected number of steps. 
    (this means the difference equation is set-up at a \textquote{higher level} than the individual vertices, a group level).
    There are 4 types of vertices: one at distance 0, 3 at distance 1, 3 at distance 2 and 1 at distance 3. We have:
    \begin{align*}
        S(0) &= 0 \\
        3 \cdot S(1) &= 3 + 1 \cdot 0 + 2 \cdot S(2) \\
        3 \cdot S(2) &= 3 + 1 \cdot S(3) + 2 \cdot  S(1) \\
        S(3) &= 1 + S(2).
    \end{align*}
    Substituting the last equation into the previous we obtain $2 S(2) = 4 + 2 S(1)$, which implies $S(1) = 7$, $S(2) = 9$ and $S(3) = 10$. \\
    \emph{My solution is way cleaner than what is on \href{https://stats.stackexchange.com/questions/139384/random-walk-on-the-edges-of-a-cube}{Stack Overflow}}!
\end{qanda}

\begin{qanda}
    \Q Consider you have two random variables, $X$ and $Y$, which are joint normal with zero means, unit variances and correlation of $rho$. Describe/derive distribution of $X$ given $Y = y$.
    \A My first approach uses the definition of conditional probability.
    Using the definition of the multivariate normal in the equation $p(x|y) \propto p(x,y)$ yields:
    \begin{equation*} 
        p(x | y) \propto \exp \left(  -\frac{1}{2} (x, y)  
            \begin{pmatrix} 1 &  \rho\\ \rho & 1 \end{pmatrix}^{-1}
            \begin{pmatrix} x \\ y \end{pmatrix} 
        \right).
        \end{equation*}
        To invert the matrix, we recall the formula for the inverse of a 2x2 matrix:
        \begin{equation*}
            \begin{pmatrix} 1 & \rho \\ \rho & 1 \end{pmatrix}^{-1} = 
            \frac{1}{1 - \rho^2}
            \begin{pmatrix} 1 & -\rho  \\ -\rho& 1 \end{pmatrix}
        \end{equation*}
        Computing the quadratic form of the matrix gives $x^2 + y^2 - 2 \rho x y$.
        This yields:
        \begin{equation*}
            p(x | y) \propto \exp \left(  
                -\frac{(x - \rho y)^2}{2(1-\rho^2)} 
            \right)
        \end{equation*}
        which is proportional to a normal distribution with mean $\rho y$ and variance $1 - \rho^2$.  In conclusion:
        \begin{equation*}
            X | Y = y \sim \text{Normal}(\rho y, 1 - \rho^2)
        \end{equation*}
        \emph{I believe this answer is correct, see \href{https://online.stat.psu.edu/stat414/lesson/21/21.1}{link}. I am not sure there is anything more elegant? In terms of generalisation, we can reduce the general case to this one through linear transformations(?).}
    \end{qanda}

