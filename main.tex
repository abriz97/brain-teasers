\documentclass{article}

\usepackage{fancyhdr}
\usepackage{amsmath}
\usepackage{natbib}
\usepackage{hyperref}
\usepackage{xcolor}
\usepackage{import}
\usepackage{csquotes}
\usepackage{tikz}

\newcommand\tsub[1]{\textsubscript{#1}}
\newcommand\tsup[1]{\textsuperscript{#1}}

\newenvironment{qanda}{\setlength{\parindent}{0pt}}{\bigskip}
\newcommand{\Q}{\bigskip\bfseries Q: }
\newcommand{\A}{\par\textbf{A:} \normalfont}

\fancyhf{}
\fancyhead[LE]{\nouppercase{\rightmark\hfill\leftmark}}
\fancyhead[RO]{\nouppercase{\leftmark\hfill\rightmark}}
\fancyfoot[LE,RO]{\hfill\thepage\hfill}
\setlength{\headheight}{24pt}
% \lhead{\thesection\sectionmark}
% \rhead{\\Page \thepage}

\pagestyle{fancy}
\pagenumbering{gobble}

\date{2024-07-04}
\title{Brain Teasers for interview preparation}


\begin{document}
  \maketitle
  Many problems are taken from \href{https://academyflex.com/wp-content/uploads/2024/03/a-practical-guide-to-quantitative-finance-interviews.pdf}{A Practical Guide to Quantitative Finance Interviews} by Xinfeng Zhou.
  \tableofcontents

  \newpage
  \pagenumbering{arabic}

  \section{Useful facts learnt along the way}

  \begin{itemize}
    \item The \href{https://en.wikipedia.org/wiki/Cayley\%E2\%80\%93Hamilton_theorem}{Cayley Hamilton theorem} states that a square matrix $A$ satisfies its own characteristic equation. Useful for finding the inverse of a matrix.
  \end{itemize}

  \import{sections/}{brainteasers.tex}
  \newpage
  \import{sections/}{calculus_linear_algebra.tex}
  \newpage
  \import{sections/}{basic_probability.tex}
  \newpage
  \bibliographystyle{plain}
  \bibliography{MyLibrary}

\end{document}
